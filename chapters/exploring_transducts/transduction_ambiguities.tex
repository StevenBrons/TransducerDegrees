\documentclass[../main.tex]{subfiles}

% \begin{document}

\section{Transition ambiguities}
By \cref{trans_spir} we have seen that we can write the transducts of a spiralling function stream as a double product:

$$\fs{f} \geq w\cdot\prod_{i=0}^{\infty}\prod_{j=0}^{l-1}x_{j}y_{j}^{\phi(i,j)}$$

for lists of words $(x_j)_{j=0}^{l-1},(y_j)_{j=0}^{l-1}\in(\2)^{l}$ a number $l>0$ a function $\phi$ and a word $w\in\2$. We want to simplify this representation without changing the degree of the transduct. It would be nice if we could replace $x_j$ by $1$ and $y_j$ by $0$. In this section we will show that this is indeed possible by showing the following equivalence:

\begin{align*}
	w\cdot\prod_{i=0}^{\infty}\prod_{j=0}^{l-1}x_{j}y_{j}^{\phi(i,j)} \equiv \prod_{i=0}^{\infty}\prod_{j=0}^{l-1}\s{10}^{\phi'(i,j)}
\end{align*}

for a different function $\phi'$ of the same form as $\phi$ and number $l'>0$.

It is easy to construct a FST that transduces the right stream into the left stream $(\leq)$. The other direction seems just as simple, but there is a difficulty here. It is possible that there are \textit{transition ambiguities}: points where a FST cannot detect the transition form $y_j^{\phi(i,j)}$ to $x_{j+1}y_{j+1}^{\phi(i,j+1)}$. We will first have to resolve these ambiguities, before we can show this equivalence.

\begin{definition}{Transition ambiguities.}\label{def_trans_amb}\\
	Let $\fs{f} \geq \sigma$ with $f\in\spir$ and $\sigma$ written as in \cref{trans_spir}:

	$$\sigma = w\cdot \prod_{i=0}^{\infty}\prod_{j=0}^{l-1}x_{j}y_{j}^{\phi(i,j)}$$

	for a number $l>0$, lists of words $(x_j)_{j=0}^{l-1},(y_j)_{j=0}^{l-1}\in(\2)^l$ a function $\phi$ of the form described in \cref{trans_spir} and a word $w\in\2$.
	Then we say that this product \textit{has transition ambiguities} if one of the following holds:
	\begin{align}
			&\exists j<l-1\qquad &y_j^{\omega} &= x_{j+1}y_{j+1}^{\omega} \label{trans_amb1.1} \\
			& &y_{l-1}^{\omega} &= x_{0}y_{0}^{\omega}  \label{trans_amb1.2} \\ 
			&\exists j<l\qquad &x_j &= \epsilon \label{trans_amb2}
	\end{align}
\end{definition}

\begin{example}{A transition ambiguity.}\label{example_trans_amb}\\
	Consider the words $u,x,y :=\s{10}$.
	Then $u^\omega=(\s{10})^\omega=\s{10}(\s{10})^\omega=xy^\omega$.
	A FST cannot recognise the transition between $u$ and $xy$ because they look ``the same'' as streams. To illustrate  this problem even more, consider the following stream with $x_0,x_1,y_0,y_1 :=\s{10}$:
	$$\sigma := \prod_{i=0}^{\infty}\prod_{j=0}^{1}x_iy_i^{\phi(i,j)} = \prod_{i=0}^{\infty}x_0y_0^{\phi(i,0)}x_1y_1^{\phi(i,1)}$$
	A FST cannot detect if $\phi(i,0)=2$ and $\phi(i,1)=3$ or $\phi(i,0)=3$ and $\phi(i,1)=2$ for some $i\geq0$.
	\begin{align*}
		x_0y_0^{2}x_1y_1^{3} =
		\s{10}(\s{10})^{2}\s{10}(\s{10})^{3} = 
		(\s{10})^{7} =
		\s{10}(\s{10})^{3}\s{10}(\s{10})^{2} = 
		x_0y_0^{3}x_1y_1^{2}
	\end{align*}
\end{example}

We will resolve ambiguities of the form given in \cref{trans_amb1.1,trans_amb1.2} in the next lemma.

\begin{lemma}\label{remove_amb_form_1}
	Let $u,x,y\in\2$ with $u,y\neq\epsilon$ and $i,j\geq0$ such that $u^i = xy^j$. Then there exists a word $v\in\2$ and $a,b\geq 0$ such that $u^mxy^n=xv^{am+bn}$ for all $m,n\geq0$.
	\begin{proof}
		We refer to \cite{streams:degrees:squares:2015}.
	\end{proof}
\end{lemma}

\begin{example}
	We use \cref{remove_amb_form_1} to resolve the transition ambiguity posed in \cref{example_trans_amb}. Let $u,x,y :=\s{10}$, then $u^2 = xy^1$. We can choose $v := \s{10}$ and $a,b := 1$. Then we have for all $m,n\geq 0$ that $(\s{10})^{m}(\s{10})(\s{10})^{n}=(\s{10})(\s{10})^{m+n}$. The essential difference is that we have merged two blocks together without changing the resulting stream. To clarify, we have merged two blocks $x_jy_j^{\phi(i,j)}$ and $x_{j+1}y_{j+1}^{\phi(i,j+1)}$ together into one new block $\tilde{x_j}\tilde{y_j}^{\phi'(i,j)}$. Note that we also need to change the function $\phi$ to a function $\phi'$ in order for this to work. That this can be done without changing the stream is shown in the \cref{remove_amb}.
\end{example}

\begin{lemma}\label{remove_amb}
	Let $f\in\spir$ and $\fs{f}\geq \sigma$. If we write $\sigma$ in the form of \cref{trans_spir}, then we can find a number $l'>0$, lists of words $(\tilde{x_j})_{j=0}^{l'-1},(\tilde{y_j})_{j=0}^{l'-1}\in(\2)^{l}$, a word $\tilde{w}\in\2$ and $\phi'$ such that 
	\begin{align*}
		\sigma = \tilde{w}\cdot\prod_{i=0}^{\infty}\prod_{j=0}^{l'-1}\tilde{x_j}\tilde{y_j}^{\phi'(i,j)}
	\end{align*}
	contains no transition ambiguities (\cref{def_trans_amb}) and $\phi'$ is of the form described in \cref{trans_spir}.
	\begin{proof}
		For this we refer to \cite{streams:degrees:squares:2015}. The idea is to iteratively remove transition ambiguities of the form given in \cref{trans_amb1.1,trans_amb1.2} by means of \cref{remove_amb_form_1}. In a similar manner ambiguities of the form given in \cref{trans_amb2} can be removed.
	\end{proof}
\end{lemma}

\begin{theorem}\label{trans_spir_10}
	Let $f\in\spir$. Then $\fs{f} \geq \sigma$ if and only if
	\begin{align*}
		\sigma \equiv \prod_{i=0}^{\infty}\prod_{j=0}^{l'-1}\s{10}^{\phi'(i,j)}
		\qquad\text{ where }\qquad
		\phi'(i,j) = \frac{f(N+il'+j) - \tilde{a_j}}{z} \in \N
	\end{align*}
	where $\tilde{a_j}\geq0$ for $j<l$ and $z,l'>0,\; N \geq 0$.
	\begin{proof}
		If $f\in\spir$ and $\fs{f} \geq \sigma$, then by \cref{trans_spir,remove_amb} we can write 
		\begin{align}
			\sigma = \tilde{w}\cdot\prod_{i=0}^{\infty}\prod_{j=0}^{l'-1}\tilde{x_j}\tilde{y_j}^{\phi'(i,j)}
			\qquad \text{ where } \qquad
			\phi'(i,j) := \frac{f(N + l'i + j) - a_j}{z} \in\N \label{trans_spir_10.pf0}
		\end{align}
		for some integer $N\geq0$, integers $z,l'>0$, a list of numbers $(\tilde{a_j})_{j=0}^{l'-1}\in(\N)^{l'-1}$, a word $\tilde{w}\in\2$ and lists of words $(\tilde{x_j})_{j=0}^{l'-1},(\tilde{y_j})_{j=0}^{l'-1}\in(\2)^{l'}$, such that \cref{trans_spir_10.pf0} contains no transition ambiguities. We need to show that $\sigma \geq \prod_{i=0}^{\infty}\prod_{j=0}^{l'-1}\s{10}^{\phi'(i,j)}$ and that $\sigma \leq \prod_{i=0}^{\infty}\prod_{j=0}^{l'-1}\s{10}^{\phi'(i,j)}$\\
		\begin{stp}{$(\geq)$}
			For this we refer to \cite{streams:degrees:squares:2015}. The idea is to find a kind of \textit{markers} to notice the transition between blocks of the stream. It is possible to find these markers because there are no transition ambiguities.
			% Let $j<l$. We will sketch the most important idea of this proof. 
			% Because \cref{trans_spir_10.pf0} contains no transition ambiguities, we know that there exists a first point $t_j\geq 0$ where $y_j^\omega(t_j) \neq x_{j + 1}y_{j + 1}^{\omega}(t_j)$ ($j+1$ is computed modulo $l$). The point $t_j$ is illustrated below. The short black lines represent words $y_j$ and $y_{j+1}$ and the red long line represents $x_{j+1}$.
			% \begin{figure}[H]
			% 	\centering
			% 	\begin{tikzpicture}[x=2cm]
			% 		\node[anchor=east] at (0,0) {$y_j^\omega = $};
			% 		\node[anchor=east] at (0,0.7) {$x_{j + 1}y_{j + 1}^{\omega} = $};
			% 		\draw[-,line width=1pt,color=red] (0.1,0.75) -- (1.2,0.75);
			% 		\foreach \i in {0,...,8}
			% 		{
			% 			\draw[-,line width=1pt] (1.2+\i/2+0.05,0.75) -- (1.2+\i/2+0.5,0.75);
			% 		}
			% 		\foreach \i in {0,...,10}
			% 		{
			% 			\draw[-,line width=1pt] (\i/2+0.05,0) -- (\i/2+0.5,0);
			% 		}

			% 		\draw[-] (1.88,-0.2) -- (1.88,1);
			% 		\node[anchor=west] at (1.80,-0.5) {$t_j$ (the first point of difference)};
			% 	\end{tikzpicture}
			% \end{figure}
			% Let $l_j,l_j'\geq 0$ be minimal such that $\abs{y_j(y_j)^{l_j}}>t_j$ and $\abs{x_{j+1}(y_{j+1})^{l_j'}} > t_j$. In the above picture, $l_j$ would be $3$ and $l_j'$ would be $2$. \\
			% Because we assumed $t_j$ to be minimal, we have that for any $\tau\in\streams$:
			% \begin{align}
			% 	y_j(y_j)^{l_j-1} &\sqsubseteq x_{j+1}(y_{j+1})^{l_j'}\tau \label{trans_spir_10.pf2} \\
			% 	y_j(y_j)^{l_j} &\not\sqsubseteq x_{j+1}(y_{j+1})^{l_j'}\tau
			% \end{align}
			% We can insert arbitrary many words $y_j^n$ in front of the right part of \cref{trans_spir_10.pf2}:
			% \begin{align}
			% 	y_j(y_j)^{l_j} \sqsubseteq (y_j)^{n}x_{j+1}(y_{j+1})^{l_j'}\tau 
			% 	\qquad \forall n>0 \label{trans_spir_10.pf3}
			% \end{align}
			% We now make sure that each block in $x_{j+1}y_{j+1}^{\phi(i,j)}$ has a prefix $x_{j+1}y_{l_j}$. Because $\fs{f}$ is spiralling, its limit tends to infinity (\cref{def_spir}.\ref{def_spir.lim}). We can easily see that this holds for $\phi$ too. So we can find a value $N\geq 0$ such that $\phi(i,j) \geq l_j$ for all $il + j \geq N$ where $j < l$.

			% Because a FST can ``save'' a bounded amount of information, we can ``look ahead'' a bounded amount of symbols (for the details, see \cite{streams:degrees:squares:2015}). By using this idea, we can construct a transducer that looks ahead and checks if the current block is about to end. This is possible due to the fact that a block end always contains $(c_j)^{l_j}$ as sub-word.
		\end{stp}
		\begin{stp}{$(\leq)$}
			We create a FST $T$ that applies a periodic encoding to $\sigma$ (see \cref{fig:trans_spir_10.pf1}). Prepending $\tilde{w}$ is possible by \cref{basic_ineq}.\ref{basic_ineq.prepend}. Define $T := (Q,q_{l'-1},\delta,\lambda)$ where $Q := \{q_0,q_1,...,q_{l'-1}\}$ and $\delta$ and $\lambda$ defined for $i<l'$ by:
			\begin{align*}
				\delta(q_i,0) &:= q_i & \lambda(q_i,0) &:= \tilde{y_i}\\
				\delta(q_i,1) &:= q_{i+1 \mod l'} & \lambda(q_i,1) &:= \tilde{x}_{i + 1\mod l'} 
			\end{align*}
			\begin{figure}[H]
				\center
				\begin{tikzpicture}
					\node[state] (q_0) at (0,0) {$q_0$};
					\node[state,right of=q_0] (q_1) {$q_1$};
					\node[state,right of=q_1] (q_2) {$q_2$};
					\node[state,initial right,right of=q_2] (q_{l-1}) {$q_{l'-1}$};

					\draw (q_0) edge node[above] {\lb{1}{\tilde{x}_1}} (q_1);
					\draw (q_1) edge node[above] {\lb{1}{\tilde{x}_2}} (q_2);
					\draw (q_2) edge[curvy] (q_{l-1});
					\draw (q_{l-1}) edge[bend left=30] node[above] {\lb{1}{\tilde{x}_0}} (q_0);

					\draw (q_0) edge[loop above] node {\lb{0}{\tilde{y_0}}} (q_0);
					\draw (q_1) edge[loop above] node {\lb{0}{\tilde{y_1}}} (q_1);
					\draw (q_2) edge[loop above] node {\lb{0}{\tilde{y_2}}} (q_2);
					\draw (q_{l-1}) edge[loop above] node {\lb{0}{\tilde{y}_{l'-1}}} (q_{l-1});
				\end{tikzpicture}
				\caption{Transducer that applies a periodic encoding to a stream.}
				\label{fig:trans_spir_10.pf1}
			\end{figure}
			Together we get the required result:
			\begin{align*}
				\prod_{i=0}^{\infty}\prod_{j=0}^{l'-1}10^{\phi'(i,j)} \geq_{T} 
				\prod_{i=0}^{\infty}\prod_{j=0}^{l'-1}\tilde{x}_j\tilde{y}_j^{\phi'(i,j)} \overset{\ref{basic_ineq}.\ref{basic_ineq.prepend}}{\geq} 
				\tilde{w}\cdot\prod_{i=0}^{\infty}\prod_{j=0}^{l'-1}\tilde{x}_j\tilde{y}_j^{\phi'(i,j)}
			\end{align*}
		\end{stp}
	\end{proof}
\end{theorem}

% \end{document}