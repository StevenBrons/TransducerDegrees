\section{Upper bounds}
We will quickly go over the existence of upper bounds of a set of streams. An easy way to create an upper bound is to \textit{zip} the set of streams. We will define two zipping functions, one that works on streams \textit{szip} and one that works on functions: \textit{\fzip}.


\begin{definition}{$\fzip$.}\\
	Let $k>0$ and $f_0,f_1,...,f_{k-1} : \N \to \N$ be natural functions.\\
	Define $\fzip_k(f_0,f_1,...,f_{k-1}):\;\N \to \N$ by
	\begin{align}
		\fzip_k(f_0,f_1,...,f_{k-1})(kn+i) = f_i(n)
		\qquad\text{ where } i<k,n\geq 0 
	\end{align}
\end{definition}

\begin{definition}{$szip$ (short for stream-zip).}\label{generalized_function_zip}\\
	Let $k>0$ and $\sigma_0,\sigma_1,...,\sigma_{k-1} \in (\streams)^k$ be natural streams.\\
	Define $szip_k(\sigma_0,\sigma_1,...,\sigma_{k-1}) \in \streams$ by
	\begin{align}
		szip_k(\sigma_0,\sigma_1,...,\sigma_{k-1}) := \fzip_k(\sigma_0,\sigma_1,...,\sigma_{k-1})
	\end{align}
\end{definition}

Although the two definitions are identical, we want to stress that we sometimes zip \textit{streams} together and sometimes zip \textit{other functions} together.

\begin{example}
	\begin{align*}
		szip_2(\0,(1)^\omega) &= (01)^\omega \\
		\fzip_2(n \mapsto 2n, n \mapsto n^2) &= 
		\begin{cases}
			n \mapsto n &\qquad \text{ if } n \text{ is even} \\
			n \mapsto \floor{\frac{n-1}{2}}^2 &\qquad \text{ if } n \text{ is odd} 
		\end{cases}
	\end{align*}
\end{example}


\begin{definition}
	If $S\of\TD$ is a set of transducer degrees. We say that $[u]\in\TD$ is an \textit{upper bound} of $S$, if for all $[\sigma]\in S$ we have $[u] \geq [\sigma]$.
\end{definition}

We can now highlight \textbf{an important difference} between the partial order on streams $(\streams,\;\geq)$ and the partial order on transducer degrees $(\streams / \equiv,\;\geq)$. It only makes sense to talk about upper bounds in the partial order of transducer degrees. Within the poset of streams, any two equivalent streams would be each others upper bound. 

\begin{lemma}
	If $\sigma,\tau\in\streams$, then $[szip_2(\sigma,\tau)]$ is an upper bound of $\{[\sigma],[\tau]\}$.
	\begin{proof}
		We show that $szip_2(\sigma,\tau) \geq_T \sigma$ and $szip_2(\sigma,\tau) \geq_{T'} \tau$ by giving FSTs $T$ and $T'$:
		\begin{align*}
			T &:= (\{q_0,q_1\},q_0,\delta,\lambda) &T' &:= (\{q_0,q_1\},q_1,\delta,\lambda) \\
			\delta(q_0,s) &:= q_1 &\lambda(q_0,s) &:= s &\qquad \text{ for } s\in\mathbf{2}\\
			\delta(q_1,s) &:= q_0 &\lambda(q_1,s) &:= \epsilon &\qquad \text{ for } s\in\mathbf{2}
		\end{align*}
		See \cref{fig:zip_trans} for a visualisation. Note that the only difference between $T$ and $T'$ is the initial state.
		\begin{figure}[H]
			\centering
			\begin{tikzpicture}
				\node[state, initial] (q0) {$q_0$};
				\node[state] (q1) at (2,0) {$q_1$};
 				\draw (q0) edge[bend left] node[above,yshift=-0.1cm] {\lb{0}{\s{0}}} node[above,yshift=0.3cm] {\lb{1}{\s{1}}} (q1);
 				\draw (q1) edge[bend left] node[below,yshift=0.1cm] {\lb{0}{\epsilon}} node[below,yshift=-0.3cm] {\lb{1}{\epsilon}} (q0);
				 
				 \node[state] (q0) at (6,0) {$q_0$};
				 \node[state, initial right] (q1) at (8,0) {$q_1$};
 				\draw (q0) edge[bend left] node[above] {\lb{0}{\s{0}}} node[above,yshift=0.4cm] {\lb{1}{\s{1}}} (q1);
 				\draw (q1) edge[bend left] node[below] {\lb{0}{\epsilon}} node[below,yshift=-0.4cm] {\lb{1}{\epsilon}} (q0);

			\end{tikzpicture}
			\caption{A FST that drops every second symbol ($T$, left) and a FST that drops every second symbol with an offset of one ($T'$, right)}
			\label{fig:zip_trans}
		\end{figure}
	\end{proof}
\end{lemma}

\begin{lemma}\label{upper_bounds_functions}
	If $f,g: \N \to \N$, then $[\fs{\fzip_2(f,g)}]$ is an upper bound of $\{[\fs{f}],[\fs{g}]\}$.
	\begin{proof}
		We need to show that $\fs{\fzip_2(f,g)} \geq \fs{f}$ and $\fs{\fzip_2(f,g)} \geq \fs{g}$. This is easily done by applying \cref{simple_fs_facts}.
		\begin{align*}
			\fs{n \mapsto \fzip_2(f,g)(n)} \overset{\ref{simple_fs_facts}.\ref{simple_fs_facts.an}}{\geq} \fs{n \mapsto \fzip_2(f,g)(2n)} = \fs{f}\\
			\fs{n \mapsto \fzip_2(f,g)(n)} \overset{\ref{simple_fs_facts}.\ref{simple_fs_facts.shift}}{\equiv} \fs{n \mapsto \fzip_2(f,g)(n + 1)} \overset{\ref{simple_fs_facts}.\ref{simple_fs_facts.an}}{\geq} \fs{n \mapsto \fzip_2(f,g)(2n + 1)} = \fs{g}
		\end{align*}
	\end{proof}
\end{lemma}

\begin{lemma}[\cite{streams:degrees:2011}]
	Every countable set of degrees $\{[\sigma_i]\}_{i=0}^{\infty}$ has an upper bound.
\end{lemma}

Now that we have seen that upper bounds exist, we can wonder if two degrees have a \textit{least upper bound} also known as the \textit{supremum}.

\begin{definition}
	A set $S\of\TD$ of transducer degrees has a \textit{least upper bound} or a \textit{supremum} $[s]\in\TD$ if, for each upper bound $[u]\in\TD$ of $S$ we have $[u] \geq [s]$.
\end{definition}

In \cref{suprema_section} we will illustrate that suprema do not always exist.