\documentclass{sthesis}
\graphicspath{ {images/}{../images/}}

% Biblatex
\usepackage[style=trad-abbrv,sorting=none,sortcites=true]{biblatex}
\addbibresource{references.bib}

\usepackage{tocloft}
\addtolength{\cftchapnumwidth}{6pt}

\widowpenalty10000
\clubpenalty10000

\begin{document}
	\begin{titlepage}
	\begin{center}
	\textsc{\LARGE Bachelor thesis\\Computing Science \&\\Mathematics}\\[1.5cm]
	\includegraphics[height=100pt]{logo}
	
	\vspace{0.4cm}
	\textsc{\Large Radboud University}\\[1cm]
	\hrule
	\vspace{0.4cm}
	\textbf{\huge Transducer Degrees}\\[0.4cm]
	\hrule
	\vspace{2cm}
	\begin{minipage}[t]{0.45\textwidth}
	\begin{flushleft} \large
	\textit{Author:}\\
	Steven Bronsveld\\
	\texttt{steven.bronsveld@student.ru.nl}\\
	s1020191
	\end{flushleft}
	\end{minipage}
	\begin{minipage}[t]{0.45\textwidth}
	\begin{flushright} \large
	\textit{Supervisor/assessor:}\\
	Prof. Dr. J.H. Geuvers\\
	\texttt{h.geuvers@cs.ru.nl}\\[1.3cm]
	\textit{Second assessor:}\\
	Prof. Dr. H. Zantema\\
	\texttt{h.zantema@cs.ru.nl}
	\end{flushright}
	\end{minipage}
	\vfill
	{\large \today}
	\end{center}
\end{titlepage}
	\begin{abstract}
		We explore the structure of transducer degrees. This is a partial order on infinite sequences known as streams. One stream $\sigma$ is considered to be greater than another stream $\tau$ if there exists a finite state transducer (FST) that transduces $\sigma$ to $\tau$. We show properties of this order and show what techniques were used to prove these properties. Our main focus is the fact that the degree of $\fs{n^2}=\s{10}^0\s{10}^1\s{10}^4\s{10}^9\s{10}^{16}\s{...}$ is an atom. This means that all transducts of $\fs{n^2}$ can be transduced back to $\fs{n^2}$ or to the zero stream $\0$.
	\end{abstract}
		
	\tableofcontents

	\begin{align*}
		&f: \N \to \N\\
		&\fs{f} := \prod_{i=0}^{\infty}\s{10}^{f(i)} = \s{10}^{f(0)}\s{10}^{f(1)}\s{10}^{f(0)}\s{10}^{f(2)}\s{...}
		% \sigma = w\cdot  \prod_{i=0}^{\infty}\prod_{j=0}^{l-1} x_j \cdot y_j^{\phi(i,j)}\\
		% \sigma \equiv \prod_{i=0}^{\infty}\prod_{j=0}^{l'-1}\s{10}^{\phi'(i,j)}\\
		% \sigma \equiv \fs{\weight\oplus(\mass\otimes \shift{k}{f})}\\
		% \gtrdot\\
		% \sigma = w \cdot x_0y_0^{\phi(0,0)}x_1y_1^{\phi(0,1)}x_0y_0^{\phi(1,0)}x_1y_1^{\phi(1,1)}...
	\end{align*}


	\chapter{Introduction}
		\subfile{chapters/introduction}

	\chapter{Preliminaries}
		\subfile{chapters/preliminaries/prerequisites}

	\newcommand{\sectionbreak}{\clearpage}

	% \chapter{Definitions}
	% 	\label{chap1}
	% 	\subfile{chapters/definitions/stream_definitions}
	% 	\subfile{chapters/definitions/fst_definitions}
	% 	\subfile{chapters/definitions/transducer_degrees}

	% \chapter{Initial investigation}
	% 	\label{chap2}
	% 	\subfile{chapters/initial_investigation/basics}
	% 	% \subfile{chapters/initial_investigation/equivalent_transducers}
	% 	\subfile{chapters/initial_investigation/funcition_streams}
	% 	\subfile{chapters/initial_investigation/upper_bounds}
	% 	\subfile{chapters/initial_investigation/atoms}

	% \chapter{Exploring transducts}
	% 	\label{chap3}
	% 	\subfile{chapters/exploring_transducts/spiralling_functions}
		% \subfile{chapters/exploring_transducts/fst_pumping_lemma}
		% \subfile{chapters/exploring_transducts/spiralling_transducts}
	% 	\subfile{chapters/exploring_transducts/transduction_ambiguities}
	% 	\subfile{chapters/exploring_transducts/mass_products}

	% \chapter{Deep results}
	% 	\label{chap4}
	% 	\subfile{chapters/deep_results/n2_atom}
	% 	\subfile{chapters/deep_results/polynomial_atoms.tex}
	% 	\subfile{chapters/deep_results/suprema_and_infima}
		
	% 	\renewcommand{\sectionbreak}{}

	% 	\chapter{Conclusion and future work}
	% 		\subfile{chapters/conclusion}
	\nocite{*}
	\printbibliography[title={References}]
\end{document}